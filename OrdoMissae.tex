\documentclass[10pt,titlepage]{article}

\usepackage[T1]{fontenc}
\usepackage[polish]{babel}
\usepackage[utf8]{inputenc}
\usepackage{lmodern}
\usepackage{enumerate}
\usepackage{color}
\selectlanguage{polish}
\title{Ordo Missae}
\author{}
\makeatletter
\def\@makechapterhead#1{%
	\vspace*{50\p@}%
	{\parindent \z@ \raggedright \normalfont
		\interlinepenalty\@M
		\Huge\bfseries  \thechapter.\quad #1\par\nobreak
		\vskip 40\p@
}}
\makeatother
\usepackage{parcolumns}
\newcommand{\latinK}[1]{\colchunk{\begin{description}\item[\textbf{S.}]{#1}
\end{description}}}
\newcommand{\latinW}[1]{\colchunk{\begin{description}\item[\textbf{M.}]{#1}
\end{description}}}
\newcommand{\polK}[1]{\colchunk{\begin{description}\item[\textbf{K.}]{#1}
	\end{description}}\colplacechunks}
\newcommand{\polW}[1]{\colchunk{\begin{description}\item[\textbf{W.}]{#1}
\end{description}}\colplacechunks}

\usepackage{geometry}
\newgeometry{tmargin=3cm, bmargin=3cm, lmargin=3cm, rmargin=3cm}

\begin{document}
	\maketitle
	\clearpage
	\tableofcontents
	\clearpage
	
	\section{MSZA KATECHUMENÓW}
	
	\subsection{CZĘŚĆ PIERWSZA}
	
	\subsubsection{Asperges - Aspersja}
	\textbf{(tylko w mszach uroczystych)}
	
	\subsubsection{Psalm wstępny}
	\textcolor{red}{Kapłan stanąwszy u stopni ołtarza, żegna się znakiem krzyża i odmawia antyfonę:}
	\begin{parcolumns}[colwidths={2=3 in},nofirstindent]{2}
		\latinK{In nomine Patris, (+) et Filii, et Spiritus Sancti. Amen.\\Introibo ad altare Dei.}
		\polK{W imię Ojca + i Syna i Ducha Świętego. Amen.\\Przystąpię do ołtarza Bożego.}
		\latinW{Ad deum qui laetificat iuventutem meam.}
		\polW{Do Boga, który jest weselem moim od młodości.}
	\end{parcolumns}
	\textbf{Psalm 42}
	\begin{parcolumns}[colwidths={2=3 in},nofirstindent]{2}
		\latinK{Iudica me Deus, et discerne causam meam de gente non sancta: ab homine iniquo et doloso erue me.}
		\polK{Bądź mi sędzia, o Boże i rozsądź sprawę moją z narodem bezbożnym; wybaw mnie od człowieka niedobrego i fałszywego.}
		\latinW{Quia tu es Deus fortitudo mea: quare me repulisti, et quare tristis incedo, dum affligit me inimicus?}
		\polW{Wszak Ty jesteś, o Boże mocą moją; czemu mnie odrzucasz i czemu smutny chodzę, gdy nieprzyjaciel mnie nęka?}
		
		\latinK{Emitte lucem tuam, et veritatem tuam: ipsa me deduxerunt, et adduxerunt in montem sanctum tuum, et in tabernacula tua.}
		\polK{Ześlij światłość Swoją i prawdę Swoją; one mnie poprowadzą i przywiodą na góre święta Twoją, aż do przybytków Twoich.}
		\latinW{Et introibo ad altare Dei: ad Deum qui laetificat iuventutem meam.}
		\polW{I przystąpię do ołtarza Bożego, do Boga, który jest weselem moim od młodości}
		
		\latinK{Confitebor tibi in cithara Deus, Deus meus: quare tristis es anima mea, et quare conturbas me?}
		\polK{Chwalić Cię będę przy dźwiękach cytry, Boże, Boże mój; czemuś smutna, duszo moja, i czemu mnie trwożysz?}
		\latinW{Spera in Deo, quoniam adhuc confitebor illi: salutare vultus mei, et Deus meus.}
		\polW{Ufaj Bogu, albowiem jeszcze uwielbiać Go będę, jako Zbawcę i Boga mego.}
		
		\latinK{Gloria Patri, et Filio, et Spiritu Sancto.}
		\polK{Chwała Ojcu i Synowi, i Duchowi Świętemu}
		\latinW{Sicut erat in principio, et nunc, et semper, et in saecula saeculorum. Amen.}
		\polW{Jak było na początku, teraz i zawsze i na wieki wieków. Amen.}
		
		\latinK{Introibo ad altare Dei}
		\polK{Przystąpię do ołtarza Bożego}
		\latinW{Ad Deum qui laetificat iuventutem meam.}
		\polW{Do Boga, który jest weselem moim od młodości}
	\end{parcolumns}	
	
	\subsubsection{Confiteor - Spowiedź powszechna}
	\begin{parcolumns}[colwidths={2=3 in},nofirstindent]{2}
		\latinK{}
		\polK{}
		\latinW{}
		\polW{}
	\end{parcolumns}
	
	
	\begin{parcolumns}[colwidths={2=3 in},nofirstindent]{2}
		\latinK{}
		\polK{}
		\latinW{}
		\polW{}
	\end{parcolumns}
	
	\begin{parcolumns}[colwidths={2=3 in},nofirstindent]{2}
		\latinK{}
		\polK{}
		\latinW{}
		\polW{}
	\end{parcolumns}
	
	
	
	
	\subsubsection{Introit}
	
	\subsection{CZĘŚĆ DRUGA}
	
	\subsubsection{Kyrie - błagalne wołanie}
	\subsubsection{Gloria - Chwała Trójcy Przenajświętszej}
	\subsubsection{Kolekta - Modlitwa Kościelna}
	
	\subsection{CZĘŚĆ TRZECIA}
	
	\subsubsection{Lekcja }
	\subsubsection{Graduał i Alleluja (względnie Traktus, Sekwencja)}
	\subsubsection{Przygotowanie do Ewangelii}
	\subsubsection{Ewangelia}
	\subsubsection{Kazanie}
	\subsubsection{Credo - Wyznanie wiary}
	
	\section{MSZA WIERNYCH}
	
	\subsection{CZĘŚĆ PIERWSZA}
	
	\subsubsection{Offertorium - Ofertorium}
	\subsubsection{Ofiarowanie Chleba}
	\subsubsection{Przygotowanie wina i wody}
	\subsubsection{Ofiarowanie wina}
	\subsubsection{Polecenie ofiar}
	\subsubsection{Lavabo - Umycie rąk}
	\subsubsection{Polecenie ofiar Trójcy Świętej}
	\subsubsection{Wezwanie do modlitwy i Sekreta}
	
	\subsection{CZĘŚĆ DRUGA}
	
	\subsubsection{Prefacja czyli Przedśpiew}
	\subsubsection{Sanctus - Święty}
	
	\subsection{KANON MSZY ŚWIĘTEJ}
	
	\subsubsection{(Pierwsza) modlitwa wstawiennicza}
	\subsubsection{Hanc Igitur - prośba o przyjęcie ofiary}
	\subsubsection{Quam Oblationem - Prośba o przeistoczenie}
	\subsubsection{Konsekracja chleba}
	\subsubsection{Konsekracja wina}
	\subsubsection{Unde Et Memores - Wspomnienie Tajemnicy Odkupienia (Anamneza)}
	\subsubsection{Supra Quae - Modlitwa o przyjęcie Ofiary bezkrwawej}
	\subsubsection{Druga modlitwa wstawiennicza}
	\subsubsection{Zakończenie}
	
	\subsection{CZĘŚĆ TRZECIA}
	
	\subsubsection{Pater Noster - Modlitwa Pańska}
	\subsubsection{Łamanie Chleba i modły o pokój}
	\subsubsection{Modlitwy przed Komunią}
	\subsubsection{Komunia kapłana}
	\subsubsection{Komunia wiernych}
	\subsubsection{Dziękczynienie}
	\subsubsection{Communio - Śpiew przy Komunii}
	\subsubsection{Postcommunio - Modlitwy po Komunii}
	\subsubsection{Placeat Tibi - Ostatnia modlitwa}
	\subsubsection{Ostatnia Ewangelia}
	
	\subsection{MODLITWY PO KAŻDEJ CICHEJ MSZY ŚWIĘTEJ}
	
\end{document}
